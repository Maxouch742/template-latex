%%-----------------------------------------------------------------%%
%%
%%     Define headers and footers for each part of the document
%%
%%-----------------------------------------------------------------%%

\fancypagestyle{plain}{
    \fancyhf{}
    \renewcommand{\headrulewidth}{0pt}
    \setlength{\headheight}{12.1638pt}
    \fancyfoot[L]{}
    \fancyfoot[R]{\thepage}
}
\fancypagestyle{preliminary}{
    \fancyhf{} % Supprimer en-tête et pied de page
    \renewcommand{\headrulewidth}{0pt} % Supprimer la ligne d'en-tête
    \renewcommand{\footrulewidth}{0pt} % Supprimer la ligne de pied de page
    \pagenumbering{gobble} % Désactiver la numérotation des pages
}
\fancypagestyle{fancy}{
    \fancyhf{} % Effacer les en-têtes et pieds de page actuels
    \renewcommand{\headrulewidth}{1pt} % Épaisseur de la ligne horizontale de l'en-tête
    \fancyhead[LE]{\leftmark} % Nom du chapitre à gauche sur les pages paires
    \fancyhead[RO]{\rightmark} % Nom du chapitre à droite sur les pages impaires
    \renewcommand{\footrulewidth}{1pt} % Épaisseur de la ligne horizontale du pied de page
    \setlength{\headheight}{12.1638pt} % Hauteur de l'en-tête
    \fancyfoot[LE,RO]{\thepage} % Numéro de page à gauche sur les pages paires, à droite sur les pages impaires
    \fancyfoot[RE,LO]{\auteur} % Auteur à droite sur les pages paires et gauche sur les pages impaires
    \fancyfoot[C]{} % Pied de page central vide
    \pagenumbering{arabic} % Numérotation en chiffres romains
}
\fancypagestyle{tablecontent}{
    \fancyhf{} % Effacer les en-têtes et pieds de page actuels
    \renewcommand{\headrulewidth}{1pt} % Épaisseur de la ligne horizontale de l'en-tête
    \fancyhead[LE]{\leftmark} % Nom du chapitre à gauche sur les pages paires
    \fancyhead[RO]{\rightmark} % Nom du chapitre à droite sur les pages impaires
    \renewcommand{\footrulewidth}{1pt} % Épaisseur de la ligne horizontale du pied de page
    \setlength{\headheight}{12.1638pt} % Hauteur de l'en-tête
    \fancyfoot[LE,RO]{\thepage} % Numéro de page à gauche sur les pages paires, à droite sur les pages impaires
    \fancyfoot[RE,LO]{\auteur} % Auteur à droite sur les pages paires et gauche sur les pages impaires
    \fancyfoot[C]{} % Pied de page central vide
    \pagenumbering{roman} % Numérotation en chiffres romains
}